\documentclass[twocolumn]{article}
\usepackage{eledmac,eledpar}
    \maxchunks{100}
    \def\endstanzaextra{\pstart\strut\skipnumbering\pend}
	\renewcommand{\Rlineflag}{}
\usepackage{fullpage}
\usepackage[dvipsnames]{xcolor}
\usepackage{sectsty}
\usepackage[french,english]{babel}
\usepackage{endnotes}
\usepackage{hyperref}
\hypersetup{%
pdftitle={Des Cannibales},%
pdfauthor={Michel de Montaigne},%
bookmarksopen={true},%
pdfdisplaydoctitle={true},%
pdfpagemode={UseOutlines},%
pdfcreator={translated by Charles Cotton; typeset by James M. Clawson},
pdfsubject={Discourse on cannibalism and culture in the Americas and Europe.},%
pdfkeywords={{noble savage}, french, world, cannibal}}

% Comment out the next five lines to avoid using XeLaTeX, or keep them in to use nice (but not-free) fonts. 
\usepackage{xltxtra,fontspec,xunicode}
    \defaultfontfeatures{Scale=MatchLowercase ,Mapping=tex-text}
    \setromanfont[Numbers=OldStyle]{Adobe Garamond Pro}
    \setsansfont{Futura Medium Italic}
    \setmonofont{Futura Condensed Medium}

\usepackage{fancyhdr}

\newcommand{\accentcolr}{\color{WildStrawberry}}
\chapterfont{\accentcolr}
\allsectionsfont{\accentcolr\sffamily}
\renewcommand{\numlabfont}{\accentcolr\sffamily\scriptsize}
\newcommand{\spotcolr}[1]{{\accentcolr#1}}

\newcommand{\secheadr}[1]{\rhead{\sffamily\spotcolr{#1}}\thispagestyle{plain}}

\newcommand{\specialbreak}{\newpage\noindent}
\newcommand{\specialbreaks}[1]{\linebreak\newpage\noindent\setline{#1}}

\title{Des cannibales \\ (Of Cannibals)}
\author{Michel de Montaigne \\ Trans. by Charles Cotton \\ \url{http://essays.quotidiana.org/montaigne/cannibals/}}
\date{1580}

\pagestyle{fancy}
\renewcommand{\sectionmark}[1]{\lhead{\spotcolr{#1}}{}}
\renewcommand{\headrulewidth}{0pt}
\addtolength{\headwidth}{\marginparsep}
\addtolength{\headwidth}{\marginparsep}
\addtolength{\headwidth}{\marginparwidth}
% \setlength{\headheight}{15pt}
\addtolength{\headsep}{\marginparsep}


\fancypagestyle{second}{%
	\fancyhead[LE,RO]{\slshape \rightmark}
	\fancyhead[LO,RE]{\slshape \leftmark}
	\fancyfoot[C]{\thepage}
	\renewcommand{\headrulewidth}{0pt}
	\renewcommand{\footrulewidth}{0pt}}

% \pagestyle{plain}

\begin{document} 
\maketitle{}

\thispagestyle{plain}
\secheadr{Des cannibales}

\begingroup
	\beginnumbering
	\autopar
	
\noindent When King Pyrrhus invaded Italy, having viewed and considered the order of the army the Romans sent out to meet him; ``I know not,'' said he, ``what kind of barbarians'' (for so the Greeks called all other nations) ``these may be; but the disposition of this army that I see has nothing of barbarism in it.''---[Plutarch, \emph{Life of Pyrrhus}, c. 8.]---As much said the Greeks of that which Flaminius brought into their country; and Philip, beholding from an eminence the order and distribution of the Roman camp formed in his kingdom by Publius Sulpicius Galba, spake to the same effect. By which it appears how cautious men ought to be of taking things upon trust from vulgar opinion, and that we are to judge by the eye of reason, and not from common report.

	I long had a man in my house that lived ten or twelve years in the New World, discovered in these latter days, and in that part of it where Villegaignon landed,---[At Brazil, in 1557.]---which he called Antarctic France. This discovery of so vast a country seems to be of very great consideration. I cannot be sure, that hereafter there may not be another, so many wiser men than we having been deceived in this. I am afraid our eyes are bigger than our bellies, and that we have more curiosity than capacity; for we grasp at all, but catch nothing but wind.

	Plato brings in Solon,---[In Timaeus.]---telling a story that he had heard from the priests of Sais in Egypt, that of old, and before the Deluge, there was a great island called Atlantis, situate directly at the mouth of the straits of Gibraltar, which contained more countries than both Africa and Asia put together; and that the kings of that country, who not only possessed that Isle, but extended their dominion so far into the continent that they had a country of Africa as far as Egypt, and extending in Europe to Tuscany, attempted to encroach even upon Asia, and to subjugate all the nations that border upon the Mediterranean Sea, as far as the Black Sea; and to that effect overran all Spain, the Gauls, and Italy, so far as to penetrate into Greece, where the Athenians stopped them: but that some time after, both the Athenians, and they and their island, were swallowed by the Flood.

	It is very likely that this extreme irruption and inundation of water made wonderful changes and alterations in the habitations of the earth, as ’tis said that the sea then divided Sicily from Italy---\thispagestyle{second}

	\begin{quote}
		\noindent\emph{Haec loca, vi quondam et vasta convulsa ruina,}\\
		\emph{Dissiluisse ferunt, quum protenus utraque tellus}\\
		\emph{Una foret}
	\end{quote}
		
	\begin{quote}
		[``These lands, they say, formerly with violence and vast desolation convulsed, burst asunder, where erewhile were.''---\emph{\AE{}neid}, iii. 414.]
	\end{quote}
	
	Cyprus from Syria, the isle of Negropont from the continent of Beeotia, and elsewhere united lands that were separate before, by filling up the channel betwixt them with sand and mud:

	\begin{quote}
		\noindent\emph{Sterilisque diu palus, aptaque remis,}\\ 
		\emph{Vicinas urbes alit, et grave sentit aratrum.}\end{quote}
	
	\begin{quote}
		[``That which was once a sterile marsh, and bore vessels on its bosom, now feeds neighbouring cities, and admits the plough.''---Horace, \emph{De Arte Poetica}, v. 65.]
	\end{quote}
	
	But there is no great appearance that this isle was this New World so lately discovered: for that almost touched upon Spain, and it were an incredible effect of an inundation, to have tumbled back so prodigious a mass, above twelve hundred leagues: besides that our modern navigators have already almost discovered it to be no island, but terra firma, and continent with the East Indies on the one side, and with the lands under the two poles on the other side; or, if it be separate from them, it is by so narrow a strait and channel, that it none the more deserves the name of an island for that.

	It should seem, that in this great body, there are two sorts of motions, the one natural and the other feverish, as there are in ours. When I consider the impression that our river of Dordogne has made in my time on the right bank of its descent, and that in twenty years it has gained so much, and undermined the foundations of so many houses, I perceive it to be an extraordinary agitation: for had it always followed this course, or were hereafter to do it, the aspect of the world would be totally changed. But rivers alter their course, sometimes beating against the one side, and sometimes the other, and some times quietly keeping the channel. I do not speak of sudden inundations, the causes of which everybody understands. In Medoc, by the seashore, the Sieur d’Arsac, my brother, sees an estate he had there, buried under the sands which the sea vomits before it: where the tops of some houses are yet to be seen, and where his rents and domains are converted into pitiful barren pasturage. The inhabitants of this place affirm, that of late years the sea has driven so vehemently upon them, that they have lost above four leagues of land. These sands are her harbingers: and we now see great heaps of moving sand, that march half a league before her, and occupy the land.

	The other testimony from antiquity, to which some would apply this discovery of the New World, is in Aristotle; at least, if that little book of \emph{Unheard of Miracles} be his---[one of the spurious publications brought out under his name---D.W.]. He there tells us, that certain Carthaginians, having crossed the Atlantic Sea without the Straits of Gibraltar, and sailed a very long time, discovered at last a great and fruitful island, all covered over with wood, and watered with several broad and deep rivers, far remote from all terra firma; and that they, and others after them, allured by the goodness and fertility of the soil, went thither with their wives and children, and began to plant a colony. But the senate of Carthage perceiving their people by little and little to diminish, issued out an express prohibition, that none, upon pain of death, should transport themselves thither; and also drove out these new inhabitants; fearing, ’tis said, lest’ in process of time they should so multiply as to supplant \specialbreak themselves and ruin their state. But this relation of Aristotle no more agrees with our new-found lands than the other.

	This man that I had was a plain ignorant fellow, and therefore the more likely to tell truth: for your better-bred sort of men are much more curious in their observation, ’tis true, and discover a great deal more; but then they gloss upon it, and to give the greater weight to what they deliver, and allure your belief, they cannot forbear a little to alter the story; they never represent things to you simply as they are, but rather as they appeared to them, or as they would have them appear to you, and to gain the reputation of men of judgment, and the better to induce your faith, are willing to help out the business with something more than is really true, of their own invention. Now in this case, we should either have a man of irreproachable veracity, or so simple that he has not wherewithal to contrive, and to give a colour of truth to false relations, and who can have no ends in forging an untruth. Such a one was mine; and besides, he has at divers times brought to me several seamen and merchants who at the same time went the same voyage. I shall therefore content myself with his information, without inquiring what the cosmographers say to the business. We should have topographers to trace out to us the particular places where they have been; but for having had this advantage over us, to have seen the Holy Land, they would have the privilege, forsooth, to tell us stories of all the other parts of the world beside. I would have every one write what he knows, and as much as he knows, but no more; and that not in this only but in all other subjects; for such a person may have some particular knowledge and experience of the nature of such a river, or such a fountain, who, as to other things, knows no more than what everybody does, and yet to give a currency to his little pittance of learning, will undertake to write the whole body of physics: a vice from which great inconveniences derive their original.

	Now, to return to my subject, I find that there is nothing barbarous and savage in this nation, by anything that I can gather, excepting, that every one gives the title of barbarism to everything that is not in use in his own country. As, indeed, we have no other level of truth and reason than the example and idea of the opinions and customs of the place wherein we live: there is always the perfect religion, there the perfect government, there the most exact and accomplished usage of all things. They are savages at the same rate that we say fruits are wild, which nature produces of herself and by her own ordi- \specialbreak nary progress; whereas, in truth, we ought rather to call those wild whose natures we have changed by our artifice and diverted from the common order. In those, the genuine, most useful, and natural virtues and properties are vigorous and sprightly, which we have helped to degenerate in these, by accommodating them to the pleasure of our own corrupted palate. And yet for all this, our taste confesses a flavour and delicacy excellent even to emulation of the best of ours, in several fruits wherein those countries abound without art or culture. Neither is it reasonable that art should gain the pre-eminence of our great and powerful mother nature. We have so surcharged her with the additional ornaments and graces we have added to the beauty and riches of her own works by our inventions, that we have almost smothered her; yet in other places, where she shines in her own purity and proper luster, she marvelously baffles and disgraces all our vain and frivolous attempts:

	\begin{quote}
		\noindent\emph{Et veniunt hederae sponte sua melius;}\\
		\emph{Surgit et in solis formosior arbutus antris;}\\
		\emph{Et volucres nulls dulcius arte canunt.}
	\end{quote}
	
	\begin{quote}
		[``The ivy grows best spontaneously, the arbutus best in shady caves; and the wild notes of birds are sweeter than art can teach.''---Propertius, i. 2, 10.]
	\end{quote}
	
	Our utmost endeavours cannot arrive at so much as to imitate the nest of the least of birds, its contexture, beauty, and convenience: not so much as the web of a poor spider.

	All things, says Plato,---[\emph{Laws}, 10.]---are produced either by nature, by fortune, or by art; the greatest and most beautiful by the one or the other of the former, the least and the most imperfect by the last.

	These nations then seem to me to be so far barbarous, as having received but very little form and fashion from art and human invention, and consequently to be not much remote from their original simplicity. The laws of nature, however, govern them still, not as yet much vitiated with any mixture of ours: but ’tis in such purity, that I am sometimes troubled we were not sooner acquainted with these people, and that they were not discovered in those better times, when there were men much more able to judge of them than we are. I am sorry that Lycurgus and Plato had no knowledge of them; for to my apprehension, what we now see in those nations, does not only surpass all the pictures with \specialbreaks{213}which the poets have adorned the golden age, and all their inventions in feigning a happy state of man, but, moreover, the fancy and even the wish and desire of philosophy itself; so native and so pure a simplicity, as we by experience see to be in them, could never enter into their imagination, nor could they ever believe that human society could have been maintained with so little artifice and human patchwork. I should tell Plato that it is a nation wherein there is no manner of traffic, no knowledge of letters, no science of numbers, no name of magistrate or political superiority; no use of service, riches or poverty, no contracts, no successions, no dividends, no properties, no employments, but those of leisure, no respect of kindred, but common, no clothing, no agriculture, no metal, no use of corn or wine; the very words that signify lying, treachery, dissimulation, avarice, envy, detraction, pardon, never heard of.---

	\begin{quote}
		[This is the famous passage which Shakespeare, through Florio’s version, 1603, or ed. 1613, p. 102, has employed in \emph{The Tempest}, ii. 1.]
	\end{quote}
	
	How much would he find his imaginary Republic short of his perfection?

	\begin{quote}
		\noindent\emph{Viri a diis recentes.}
	\end{quote}
	
	\begin{quote}
		[``Men fresh from the gods.''---Seneca, \emph{Ep.}, 90.]
	\end{quote}

	\begin{quote}
		\noindent\emph{Hos natura modos primum dedit.}
	\end{quote}
	
	\begin{quote}
		[``These were the manners first taught by nature.'' ---Virgil, \emph{Georgics}, ii. 20.]
	\end{quote}
	
	As to the rest, they live in a country very pleasant and temperate, so that, as my witnesses inform me, ’tis rare to hear of a sick person, and they moreover assure me, that they never saw any of the natives, either paralytic, bleareyed, toothless, or crooked with age. The situation of their country is along the sea-shore, enclosed on the other side towards the land, with great and high mountains, having about a hundred leagues in breadth between. They have great store of fish and flesh, that have no resemblance to those of ours: which they eat without any other cookery, than plain boiling, roast- \specialbreaks{253}ing, and broiling. The first that rode a horse thither, though in several other voyages he had contracted an acquaintance and familiarity with them, put them into so terrible a fright, with his centaur appearance, that they killed him with their arrows before they could come to discover who he was. Their buildings are very long, and of capacity to hold two or three hundred people, made of the barks of tall trees, reared with one end upon the ground, and leaning to and supporting one another at the top, like some of our barns, of which the covering hangs down to the very ground, and serves for the side walls. They have wood so hard, that they cut with it, and make their swords of it, and their grills of it to broil their meat. Their beds are of cotton, hung swinging from the roof, like our seamen’s hammocks, every man his own, for the wives lie apart from their husbands. They rise with the sun, and so soon as they are up, eat for all day, for they have no more meals but that; they do not then drink, as Suidas reports of some other people of the East that never drank at their meals; but drink very often all day after, and sometimes to a rousing pitch. Their drink is made of a certain root, and is of the colour of our claret, and they never drink it but lukewarm. It will not keep above two or three days; it has a somewhat sharp, brisk taste, is nothing heady, but very comfortable to the stomach; laxative to strangers, but a very pleasant beverage to such as are accustomed to it. They make use, instead of bread, of a certain white compound, like coriander seeds; I have tasted of it; the taste is sweet and a little flat. The whole day is spent in dancing. Their young men go a-hunting after wild beasts with bows and arrows; one part of their women are employed in preparing their drink the while, which is their chief employment. One of their old men, in the morning before they fall to eating, preaches to the whole family, walking from the one end of the house to the other, and several times repeating the same sentence, till he has finished the round, for their houses are at least a hundred yards long. Valour towards their enemies and love towards their wives, are the two heads of his discourse, never failing in the close, to put them in mind, that ’tis their wives who provide them their drink warm and well seasoned. The fashion of their beds, ropes, swords, and of the wooden bracelets they tie about their wrists, when they go to fight, and of the great canes, bored hollow at one end, by the sound of which they keep the cadence of their dances, are to be seen in several places, and amongst others, at my house. They shave all over, and much more neatly than we, without other razor than one of wood or stone. They believe in the immor-\specialbreak tality of the soul, and that those who have merited well of the gods are lodged in that part of heaven where the sun rises, and the accursed in the west.

	They have I know not what kind of priests and prophets, who very rarely present themselves to the people, having their abode in the mountains. At their arrival, there is a great feast, and solemn assembly of many villages: each house, as I have described, makes a village, and they are about a French league distant from one another. This prophet declaims to them in public, exhorting them to virtue and their duty: but all their ethics are comprised in these two articles, resolution in war, and affection to their wives. He also prophesies to them events to come, and the issues they are to expect from their enterprises, and prompts them to or diverts them from war: but let him look to’t; for if he fail in his divination, and anything happen otherwise than he has foretold, he is cut into a thousand pieces, if he be caught, and condemned for a false prophet: for that reason, if any of them has been mistaken, he is no more heard of.

	Divination is a gift of God, and therefore to abuse it, ought to be a punishable imposture. Amongst the Scythians, where their diviners failed in the promised effect, they were laid, bound hand and foot, upon carts loaded with firs and bavins, and drawn by oxen, on which they were burned to death.---[Herodotus, iv. 69.]---Such as only meddle with things subject to the conduct of human capacity, are excusable in doing the best they can: but those other fellows that come to delude us with assurances of an extraordinary faculty, beyond our understanding, ought they not to be punished, when they do not make good the effect of their promise, and for the temerity of their imposture?

	They have continual war with the nations that live further within the mainland, beyond their mountains, to which they go naked, and without other arms than their bows and wooden swords, fashioned at one end like the head of our javelins. The obstinacy of their battles is wonderful, and they never end without great effusion of blood: for as to running away, they know not what it is. Every one for a trophy brings home the head of an enemy he has killed, which he fixes over the door of his house. After having a long time treated their prisoners very well, and given them all the regales they can think of, he to whom the prisoner belongs, invites a great assembly of his friends. They being come, he ties a rope to one of the arms of the prisoner, of which, at a distance, out of his reach, he holds the one end himself, and gives to the friend he loves best the other arm to hold after the same manner; which being, done, \specialbreaks{353}they two, in the presence of all the assembly, dispatch him with their swords. After that, they roast him, eat him amongst them, and send some chops to their absent friends. They do not do this, as some think, for nourishment, as the Scythians anciently did, but as a representation of an extreme revenge; as will appear by this: that having observed the Portuguese, who were in league with their enemies, to inflict another sort of death upon any of them they took prisoners, which was to set them up to the girdle in the earth, to shoot at the remaining part till it was stuck full of arrows, and then to hang them, they thought those people of the other world (as being men who had sown the knowledge of a great many vices amongst their neighbours, and who were much greater masters in all sorts of mischief than they) did not exercise this sort of revenge without a meaning, and that it must needs be more painful than theirs, they began to leave their old way, and to follow this. I am not sorry that we should here take notice of the barbarous horror of so cruel an action, but that, seeing so clearly into their faults, we should be so blind to our own. I conceive there is more barbarity in eating a man alive, than when he is dead; in tearing a body limb from limb by racks and torments, that is yet in perfect sense; in roasting it by degrees; in causing it to be bitten and worried by dogs and swine (as we have not only read, but lately seen, not amongst inveterate and mortal enemies, but among neighbours and fellow-citizens, and, which is worse, under colour of piety and religion), than to roast and eat him after he is dead.

	Chrysippus and Zeno, the two heads of the Stoic sect, were of opinion that there was no hurt in making use of our dead carcasses, in what way soever for our necessity, and in feeding upon them too;---[Diogenes Laertius, vii. 188.]---as our own ancestors, who being besieged by Caesar in the city Alexia, resolved to sustain the famine of the siege with the bodies of their old men, women, and other persons who were incapable of bearing arms.

	\begin{quote}
		\noindent\emph{Vascones, ut fama est, alimentis talibus usi}\\
		\emph{Produxere animas.}
	\end{quote}
	
	\begin{quote}
		[``'Tis said the Gascons with such meats appeased their hunger.''---Juvenal, Sat., xv. 93.]
	\end{quote}

	And the physicians make no bones of employing it to all sorts of use, either to apply it outwardly; or to give\specialbreaks{398}it inwardly for the health of the patient. But there never was any opinion so irregular, as to excuse treachery, disloyalty, tyranny, and cruelty, which are our familiar vices. We may then call these people barbarous, in respect to the rules of reason: but not in respect to ourselves, who in all sorts of barbarity exceed them. Their wars are throughout noble and generous, and carry as much excuse and fair pretence, as that human malady is capable of; having with them no other foundation than the sole jealousy of valour. Their disputes are not for the conquest of new lands, for these they already possess are so fruitful by nature, as to supply them without labour or concern, with all things necessary, in such abundance that they have no need to enlarge their borders. And they are, moreover, happy in this, that they only covet so much as their natural necessities require: all beyond that is superfluous to them: men of the same age call one another generally brothers, those who are younger, children; and the old men are fathers to all. These leave to their heirs in common the full possession of goods, without any manner of division, or other title than what nature bestows upon her creatures, in bringing them into the world. If their neighbours pass over the mountains to assault them, and obtain a victory, all the victors gain by it is glory only, and the advantage of having proved themselves the better in valour and virtue: for they never meddle with the goods of the conquered, but presently return into their own country, where they have no want of anything necessary, nor of this greatest of all goods, to know happily how to enjoy their condition and to be content. And those in turn do the same; they demand of their prisoners no other ransom, than acknowledgment that they are overcome: but there is not one found in an age, who will not rather choose to die than make such a confession, or either by word or look recede from the entire grandeur of an invincible courage. There is not a man amongst them who had not rather be killed and eaten, than so much as to open his mouth to entreat he may not. They use them with all liberality and freedom, to the end their lives may be so much the dearer to them; but frequently entertain them with menaces of their approaching death, of the torments they are to suffer, of the preparations making in order to it, of the mangling their limbs, and of the feast that is to be made, where their carcass is to be the only dish. All which they do, to no other end, but only to extort some gentle or submissive word from them, or to frighten them so as to make them run away, to\specialbreaks{447}obtain this advantage that they were terrified, and that their constancy was shaken; and indeed, if rightly taken, it is in this point only that a true victory consists:

	\begin{quote}
		\noindent\emph{Victoria nulla est,}\\
			\emph{Quam quae confessor animo quoque subjugat hostes.}
	\end{quote}
		
	\begin{quote}
		[``No victory is complete, which the conquered do not admit to be so.''---Claudius, \emph{De Sexto Consulatu Honorii}, v. 248.]
	\end{quote}
	
	The Hungarians, a very warlike people, never pretend further than to reduce the enemy to their discretion; for having forced this confession from them, they let them go without injury or ransom, excepting, at the most, to make them engage their word never to bear arms against them again. We have sufficient advantages over our enemies that are borrowed and not truly our own; it is the quality of a porter, and no effect of virtue, to have stronger arms and legs; it is a dead and corporeal quality to set in array; ’tis a turn of fortune to make our enemy stumble, or to dazzle him with the light of the sun; ’tis a trick of science and art, and that may happen in a mean base fellow, to be a good fencer. The estimate and value of a man consist in the heart and in the will: there his true honour lies. Valour is stability, not of legs and arms, but of the courage and the soul; it does not lie in the goodness of our horse or our arms but in our own. He that falls obstinate in his courage---

	\begin{quote}
		\noindent\emph{Si succiderit, de genu pugnat}
	\end{quote}
	
	\begin{quote}
		[``If his legs fail him, he fights on his knees.''---Seneca, \emph{De Providentia}, c. 2.]
	\end{quote}
	
\noindent{}---he who, for any danger of imminent death, abates nothing of his assurance; who, dying, yet darts at his enemy a fierce and disdainful look, is overcome not by us, but by fortune; he is killed, not conquered; the most valiant are sometimes the most unfortunate. There are defeats more triumphant than victories. Never could those four sister victories, the fairest the sun ever be held, of Salamis, Plataea, Mycale, and Sicily, venture to oppose all their united glories, to the single glory of the discomfiture of King Leonidas and his men, at the pass of Thermopylae. Who ever ran with a more glorious desire and greater ambition, to the winning, than Captain \specialbreak Iscolas to the certain loss of a battle?---[Diodorus Siculus, xv. 64.]---Who could have found out a more subtle invention to secure his safety, than he did to assure his destruction? He was set to defend a certain pass of Peloponnesus against the Arcadians, which, considering the nature of the place and the inequality of forces, finding it utterly impossible for him to do, and seeing that all who were presented to the enemy, must certainly be left upon the place; and on the other side, reputing it unworthy of his own virtue and magnanimity and of the Lacedaemonian name to fail in any part of his duty, he chose a mean betwixt these two extremes after this manner; the youngest and most active of his men, he preserved for the service and defence of their country, and sent them back; and with the rest, whose loss would be of less consideration, he resolved to make good the pass, and with the death of them, to make the enemy buy their entry as dear as possibly he could; as it fell out, for being presently environed on all sides by the Arcadians, after having made a great slaughter of the enemy, he and his were all cut in pieces. Is there any trophy dedicated to the conquerors which was not much more due to these who were overcome? The part that true conquering is to play, lies in the encounter, not in the coming off; and the honour of valour consists in fighting, not in subduing.

	But to return to my story: these prisoners are so far from discovering the least weakness, for all the terrors that can be represented to them, that, on the contrary, during the two or three months they are kept, they always appear with a cheerful countenance; importune their masters to make haste to bring them to the test, defy, rail at them, and reproach them with cowardice, and the number of battles they have lost against those of their country. I have a song made by one of these prisoners, wherein he bids them ``come all, and dine upon him, and welcome, for they shall withal eat their own fathers and grandfathers, whose flesh has served to feed and nourish him. These muscles,'' says he, ``this flesh and these veins, are your own: poor silly souls as you are, you little think that the substance of your ancestors’ limbs is here yet; notice what you eat, and you will find in it the taste of your own flesh:'' in which song there is to be observed an invention that nothing relishes of the barbarian. Those that paint these people dying after this manner, represent the prisoner spitting in the faces of his executioners and making wry mouths at them. And ’tis most certain, that to the very last gasp, they never cease to brave and defy them both in word and gesture. In plain truth, these men are very savage\specialbreaks{539}in comparison of us; of necessity, they must either be absolutely so or else we are savages; for there is a vast difference betwixt their manners and ours.

	The men there have several wives, and so much the greater number, by how much they have the greater reputation for valour. And it is one very remarkable feature in their marriages, that the same jealousy our wives have to hinder and divert us from the friendship and familiarity of other women, those employ to promote their husbands’ desires, and to procure them many spouses; for being above all things solicitous of their husbands’ honour, ’tis their chiefest care to seek out, and to bring in the most companions they can, forasmuch as it is a testimony of the husband’s virtue. Most of our ladies will cry out, that ’tis monstrous; whereas in truth it is not so, but a truly matrimonial virtue, and of the highest form. In the Bible, Sarah, with Leah and Rachel, the two wives of Jacob, gave the most beautiful of their handmaids to their husbands; Livia preferred the passions of Augustus to her own interest; ---[Suetonius, \emph{Life of Augustus}, c. 71.]---and the wife of King Deiotarus, Stratonice, did not only give up a fair young maid that served her to her husband’s embraces, but moreover carefully brought up the children he had by her, and assisted them in the succession to their father’s crown.

	And that it may not be supposed, that all this is done by a simple and servile obligation to their common practice, or by any authoritative impression of their ancient custom, without judgment or reasoning, and from having a soul so stupid that it cannot contrive what else to do, I must here give you some touches of their sufficiency in point of understanding. Besides what I repeated to you before, which was one of their songs of war, I have another, a love-song, that begins thus:

	\begin{quote}
		``Stay, adder, stay, that by thy pattern my sister may draw the fashion and work of a rich ribbon, that I may present to my beloved, by which means thy beauty and the excellent order of thy scales shall for ever be preferred before all other serpents.''
	\end{quote}
	
	Wherein the first couplet, ``Stay, adder,'' \&c., makes the burden of the song. Now I have conversed enough with poetry to judge thus much that not only there is nothing barbarous in this invention, but, moreover, that it is perfectly Anacreontic. To which it may be added, that their language is soft, of a pleasing accent, and something bordering upon the Greek termination.

	Three of these people, not foreseeing how dear their knowledge of the corruptions of this part of the world will one day cost their happiness and repose, and that the effect of this commerce will be their ruin, as I presuppose it is in a very fair way (miserable men to suffer themselves to be deluded with desire of novelty and to have left the serenity of their own heaven to come so far to gaze at ours!), were at Rouen at the time that the late King Charles IX. was there. The king himself talked to them a good while, and they were made to see our fashions, our pomp, and the form of a great city. After which, some one asked their opinion, and would know of them, what of all the things they had seen, they found most to be admired? To which they made answer, three things, of which I have forgotten the third, and am troubled at it, but two I yet remember. They said, that in the first place they thought it very strange that so many tall men, wearing beards, strong, and well armed, who were about the king (’tis like they meant the Swiss of the guard), should submit to obey a child, and that they did not rather choose out one amongst themselves to command. Secondly (they have a way of speaking in their language to call men the half of one another), that they had observed that there were amongst us men full and crammed with all manner of commodities, whilst, in the meantime, their halves were begging at their doors, lean and half-starved with hunger and poverty; and they thought it strange that these necessitous halves were able to suffer so great an inequality and injustice, and that they did not take the others by the throats, or set fire to their houses.

	I talked to one of them a great while together, but I had so ill an interpreter, and one who was so perplexed by his own ignorance to apprehend my meaning, that I could get nothing out of him of any moment: Asking him what advantage he reaped from the superiority he had amongst his own people (for he was a captain, and our mariners called him king), he told me, to march at the head of them to war. Demanding of him further how many men he had to follow him, he showed me a space of ground, to signify as many as could march in such a compass, which might be four or five thousand men; and putting the question to him whether or no his authority expired with the war, he told me this remained: that when he went to visit the villages of his dependence, they planed him paths through the thick of their woods, by which he might pass at his ease. All this does not sound very ill, and the last was not at all amiss, for they wear no breeches.
	
	\endnumbering
\endgroup



% \newpage
% \begingroup
% \parindent 0pt
% \parskip 2ex
% \def\enotesize{\normalsize}
% \theendnotes
% \endgroup

\end{document}